\subsection{Estructura financiera}

Este pilar es el núcleo del plan de negocio, se obtiene información vital acerca de la
viabilidad del proyecto, integra las finanzas pertinentes para generar el proyecto, y se fundamenta en 6 puntos básicos.

\begin{itemize}
    \item Estados de resultados pro-forma proyectados a tres años: Se presenta una visión del comportamiento del negocio,teniendo en cuenta la estructura financiera. “Se calcula considerando las siguientes variables: cuántas unidades y a qué precio se venderán, costo de ventas por unidad, costos fijos, costos variables, intereses (posesión de créditos) e impuestos. El resultado será la utilidad neta”. Se establecerá por medio de ese estudio.

    \item Balance general proyectado a cinco años: Esquema estructurado en 2 variables: Lo que compone la empresa y cómo se financiará. Contempla desde mobiliario y equipo (activos de la compañía), así como de dónde surgieron los recursos para adquirirlos.

    \item Flujo de caja pro-forma proyectado a cinco años: En este punto definimos las políticas de cuentas por cobrar, el ciclo de venta, el objetivo es que este reporte responda a las siguientes preguntas: ¿Cuándo se va a solicitar capital?, ¿De donde se obtendrán este capital?.

    \item Análisis del punto de equilibrio: Índice de las unidades de productos o servicios que la empresa debe vender para cubrir los costos fijos de la operación, dato relevante para    determinar el momento en que las ventas inician a generar ganancias y utilidades a la   compañía.
\end{itemize}