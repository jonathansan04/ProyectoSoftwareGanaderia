\subsection{Entorno}

En este pilar se consolida la organización en el mercado, realizando el análisis del entorno, integrado por la competencia y aliados, y los potenciales clientes.

\begin{itemize}
    \item Matriz DOFA: Metodología centrada en proveer información de un escenario en el que la empresa se encuentra en el mercado, bajo 4 perfiles, Debilidades, Oportunidades, Fortalezas y Amenazas.
    \item Descripción del publico objetivo: Ofrece la presentación y análisis de la población objetivo de la oferta de valor de la empresa, utilizando estadísticas útiles y así centralizar la información.
    \item Investigación demográfica del mercado: Se profundiza en el crecimiento del mercado en un periodo determinado hasta el presente, se tienen en cuenta el estado actual del mercado y cómo se proyecta a corto, mediano y largo plazo.
    \item Frecuencia de adquisición del producto: Define el cálculo de un índice de aproximación para determinar la frecuencia posible en que los clientes potenciales accedan al producto ofrecido, obteniendo proyecciones con un margen de error reducido.
    \item Estudio de la competencia: Basado en el estudio del Benchmarking, se establece la idea y los pilares de cada ente analizado, teniendo en cuenta como puntos primordiales el valor agregado de la competencia y la estrategia de mercadotecnia usada.
\end{itemize}