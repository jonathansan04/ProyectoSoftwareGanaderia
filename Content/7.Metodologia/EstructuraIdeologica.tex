\subsection{Estructura ideológica}

En este pilar se ubica el desarrollo del fundamento conceptual de la empresa, (EMPRENDEDORES., 2019) siendo esta la carta de presentación, conformada por:

\begin{itemize}
    \item Nombre de la empresa: Se define el nombre de la organización, con una breve descripción y el porqué de la elección.
    
    \item Misión: La misión debe abarcar el fundamento de creación de la empresa, adicionalmente debe integrar las filosofía, valores e intereses de los integrantes de la empresa, enfoque al cliente, producto y/o servicio, políticas y tecnologías.
    
    \item Visión: La visión está conformada por la proyección de la empresa en unos años, incluyendo la dirección del negocio y los stakeholders, descripción del estado futuro de la empresa especificando el tiempo.
    
    \item Valores: Este punto establece las reglas bajo las cuales está regida la organización, la relación con los clientes, trabajadores y la sociedad en general.
    
    \item Ventajas competitivas: Da una descripción de los puntos fuertes del producto que se ofrece, dando claridad en su utilidad y valor agregado frente a la competencia.
\end{itemize}