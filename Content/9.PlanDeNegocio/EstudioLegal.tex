\subsection{Estudio legal}

El estudio legal de este proyecto pretende determinar la viabilidad ante las normas y leyes que rigen en la constitución colombiana, reconociendo la empresa, los alcances, limitaciones y obligaciones que estén relacionados con la naturaleza del mismo, se busca definir en pocas palabras su formalización ante las entidades estatales.

\subsubsection*{Tipo de sociedad}

Hire.it será constituido bajo la sociedad por acciones simplificadas (S.A.S). Sociedad
creada en la legislación colombiana por la ley 1258 de 2008, quienes para efectos tributarios, se rigen por las reglas aplicables a las sociedades anónimas.

Este tipo de sociedad es la más conveniente con respecto al tema tributario. Desde su entrada en vigencia, con la ley 1258 de 2008, el 54\%& de las empresas que se han creado en Colombia lo han hecho bajo esta figura. Esta modalidad permite a los emprendedores simplificar trámites y comenzar su proyecto con un bajo presupuesto. Por ejemplo, las SAS no necesitan revisor fiscal y pueden ser personas naturales o jurídicas.

\newline
\textbf{Especificaciones generales}
\newline

La empresa será construida bajo las siguientes pautas:

\begin{itemize}
    \item \textbf{Socios : } Jorge Andrés Rojas Bautista con Cédula de Ciudadanía No 1031178177 y Daniela Alexandra Martinez Rueda No 1000251664
    \item \textbf{Nombre de la empresa :} Hire.IT S.A.S
    \item \textbf{Duración :} Tiempo indefinida.
    \item \textbf{Objeto social :} Empresa cuyo objeto se encuentra en el desarrollo e innovación de herramientas tecnológicas.
    \item \textbf{Responsabilidad sobre los aportes :} 50/50 entre socios.
    \item \textbf{Representante legal: } Jorge Andrés Rojas Bautista
\end{itemize}

\subsubsection*{Obligaciones legales}

Las obligaciones legales que serán presentadas a continuación corresponden en el marco legal de la ley 158 de 2008 “por medio de la cual se crea la Sociedad por Acciones Simplificada”

\begin{enumerate}
    \item Asamblea Ordinaria de Accionistas por lo menos una vez al año.
    \item Renovación de la matrícula mercantil dentro de los primeros tres meses del año en la Cámara de Comercio.
    \item Firma electrónica de los representantes legales, resolución 070 de la DIAN.
    \item Agentes de retención en la fuente a título de Renta, Iva, Ica, etc.
    \item Están obligados a expedir facturas.
    \item Gravamen a los movimientos financieros.
    \item Están obligados a llevar la contabilidad de su empresa.
    \item Están obligados a tener revisor fiscal según el monto de sus ingresos o activos.
\end{enumerate}

\subsubsection*{Proceso para disolución de la empresa}

Para la disolución de una empresa constituida bajo Sociedad por Acciones Simplificadas se harán los siguientes pasos:

\begin{enumerate}
    \item En el caso de las SAS, Empresa Unipersonal y Sociedades de la Ley 1014 de 2006, la declaratoria de disolución por mutuo acuerdo podrá realizarse por acta o documento privado.
    
    \item Realizar el registro del acta de disolución en Cámara de Comercio. A partir del registro, la sociedad aparecerá con el apellido “en liquidación”. Se deberá entregar copia del acta en la Cámara de Comercio, la original deberá quedarse en el archivo de la sociedad.
    
    \item Reporte a la Oficina de Cobranzas de la DIAN sobre deudas fiscales de la Sociedad. El liquidador deberá hacerlo dentro de los diez (10) días siguientes al registro de la disolución en la Cámara de Comercio..
    
    \item Emitir avisos que informen que la sociedad se encuentra en trámite de liquidación.
    
    \item Elaboración de inventario del patrimonio social y balance final de la sociedad. Este paso debe realizarlo el liquidador.
    
    \item Pagar pasivo externo. Esto también debe hacerlo el liquidador. Asimismo, deberá realizar el pago de las obligaciones fiscales y efectuar la declaración de renta final.
    
    \item Distribuir remanentes entre socios o accionistas. Por parte del liquidador.
    
    \item Elaborar el proyecto de liquidación. Por parte del liquidador. Se debe tener en cuenta que la cuenta final de liquidación, como mínimo, debe contener:
    
    \begin{itemize}
        \item Inventarios.
        \item Balance general.
        \item Estado de pérdidas y ganancias.
        \item Pasivos de la entidad.
        \item Se solicita el estado de cuenta a la DIAN.
        \item Pago de pasivos.
        \item Indicación del remanente.
        \item Destinación del remanente.
    \end{itemize}
    
    \item Realizar una reunión de Junta de Socios o Asamblea de Accionistas para aprobar el proyecto de liquidación.
    
    \item Realizar el registro del acta de la cuenta final de liquidación ante la Cámara de Comercio se deberá cancelar una tarifa de impuesto de registro del 0.7\% sobre el valor de los remanentes de la empresa después de pagar su pasivo externo. En el caso de que no haya remanente a repartir se pagará como un acto sin cuantía.
    
\end{enumerate}

\subsubsection*{Propiedad intelectual}

De acuerdo al decreto 1360 de 1989 artículo 1, presenta lo siguiente: “De conformidad con lo previsto en la Ley 23 de 1982 sobre Derechos de Autor, el soporte lógico (software) se considera como creación propia del dominio literario”. Por tal motivo todo software se ha de desarrollar deberá ser sometido a procesos de registro por obra, el cual será suministrado por la dirección nacional de derechos de autor.