\section{Conclusiones}

Después de realizar el análisis de la situación económica que se provoco por la pandemia COVID-19, se observaron las graves distorsiones a la oferta y demanda en diferentes áreas de mercado, especialmente en los países dependientes del comercio internacional, exportaciones y financiamiento externo. Por un lado, la pandemia puso en evidencia la alta exposición y vulnerabilidad del país a diversos choques domésticos como externos. Por otro lado, se presentó una oportunidad para  consolidar a Colombia mediante cambios  económicos,  sociales  e  institucionales profundos \cite{desafios} que  colaboren  a  un  crecimiento  de  mediano  plazo  sólido,  inclusivo  y  sostenible  a otras actividades económicas.

Una de las actividades que se implanto y creció para quedarse en Colombia es el trabajo remoto en áreas de tecnología.El aumento del teletrabajo ha dado lugar a algunos desarrollos interesantes en los salarios tecnológicos, y estas tendencias no muestran signos de detenerse. De acuerdo a la información obtenida en los antecedentes seleccionando la capital Bogotana, fue posible tomar en cuenta la población mas densa en términos de vacantes tecnológicas, de manera que, en nuestro análisis e investigación pudimos ver en términos cuantitativos cuantas  candidaturas aproximadamente de aspirantes en la ciudad existen, de forma que, nuestra problemática fuera relevante para la población que escogimos con énfasis a ingeniería web. 

Teniendo en cuenta lo anterior, se abordo la problemática por medio de un análisis y presentación del modelo de negocio CANVA para la conformación de una empresa donde fue posible evidenciar los beneficios de esta metodología al ser un instrumento cualitativo. La facilidad de esta herramienta nos permitió complementar la información expuesta con otros modelos cuantitativos como lo son el análisis técnico y financiero evidenciado en las secciones anteriores.

Por otro lado, luego de realizar las evaluaciones correspondientes a la matriz PEYEA es posible determinar que nos encontramos en el cuadrante intensivo/agresivo.  De acuerdo a los resultados es posible implementar  mejores estrategias en su posición interna y externa para empezar a competir en nuevos cuadrantes de la competencia, así mismo podemos aprovechar la penetración, desarrollo del mercado y producto para hacer una integración en todas las direcciones con intención de mejora.

También observando los resultados de la matriz DOFA es posible determinar las fortalezas y debilidades que tiene el plan de negocio frente a la competencia, se hizo un enfoque en los factores mas importantes que pueden afectar el negocio con el fin de tener un entendimiento de los posibles riesgos a futuro que se enfrentaran.

Por ultimo, fue posible proyectar la propuesta de prototipo mínimo viable mediante una plataforma web aplicando las estrategias gamificadas en el proceso automatizado de preselección técnica en Ingeniería Web. De igual forma contiene la funcionalidad especificada en la sección de casos de uso y requerimientos siguiendo patrones arquitectónicos de diseño software y las tecnologías mas recientes para la elaboración del mismo, cumpliendo con los requisitos de calidad de software que exige la carrera de Ingeniería de sistemas.

