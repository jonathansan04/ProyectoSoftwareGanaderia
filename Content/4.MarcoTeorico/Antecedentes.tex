\subsection{Antecedentes}

\subsubsection{Nivel Regional}

Dado los importantes avances del sector de Software y TI, el desarrollo de diversos programas para combatir el desperdicio de alimentos haciendo uso de la tecnología es cada vez más frecuente, El Ministerio de Ambiente y Desarrollo Sostenible de Colombia ha sido uno de los interesados en asistir esta problemática , el 29 de septiembre del 2022, en medio de la conmemoración del Día Internacional de Concienciación sobre la Pérdida y el Desperdicio de Alimentos resaltó el plan de acción llevado, que consiste en la realización de campañas y decretos para construir una política que contribuya a mejorar la situación.\cite{MinAmbiente}

Por otro lado, la Sociedad de Agricultores de Colombia (SAC), el Banco de Desarrollo de América Latina (CAF) y la Organización de las Naciones Unidas para la Alimentación y la Agricultura (FAO) se unieron para compartir con productores agropecuarios y otros actores de la cadena alimentaria, soluciones para reducir la pérdida y desperdicio de alimentos, desde la producción hasta el consumo, aplicando una estratégia de formación virtual donde brindan herramientas y cursos de preparación a los sectores involucrados. \cite{NUC}

En este punto se visibiliza la oportunidad de encontrar una solución al problema desde la tecnología y los nómadas digitales, ultimamente han surgido aplicaciones móviles con el mismo objetivo como Last Food, Eat'n Save y Planet Oliver, abriendo campo al desarrollo de productos tecnológicos para la disminución del desperdicio de alimentos. 

\subsubsection{Nivel Global}

En medio de la era Post-Covid se involucró de manera acelerada distintas tecnologías para generar soluciones a los diferentes problemas, Jork, la aplicación de domicilios y Olio, la aplicación de intercambio gratuito anunciaron una iniciativa para luchar contra el desperdicio de alimentos en México \cite{JOlio}, donando a comunidades vulnerables comida que no se ha vendido pero apta para el consumo.

Al mismo tiempo en España nace \textit{Encantado de Comerte}, una empresa en la que se publican lotes de alimentos que los comercios no han logrado vender a lo largo del día con un descuento mínimo del 50\%\cite{EnDComerte}, todo manejado por medio de una aplicación. Iniciativas como las anteriores no solo genera beneficios económicos sino también sociales, significando una ventaja tanto para los comerciantes como para las personas de bajos ingresos gracias a la aplicación de las TIC. 
