 %esto es lo nuevo que agregue
\phantomsection
\section*{Resumen}
\addcontentsline{toc}{section}{Resumen}
\noindent 

% Motivo del documento, enfasis en el plan de negocios, descripcion de los procesos y resultados esperados, metodologias y trabajo realizado (100 - 150 palabras)



El documento tiene como objetivo presentar el plan de negocios para la creación de una plataforma que apoye el proceso de selección de personal profesional en TI enfocado en la ingeniería web usando técnicas de gamificación denominada HIRE.IT en la ciudad de Bogotá. Para una correcta estructuración del plan de negocios se creo un modelo de negocio a partir de la metodología canvas en donde la propuesta de valor se enfocó en hacer uso de estrategias de gamificación para optimizar los tiempos y calidad de los procesos de evaluación  técnica para los perfiles especializados en ingeniería web. Por otra parte, es posible evidenciar el análisis PEYEA/DOFA donde se identificó y evaluó las fortalezas y posicionamiento de la propuesta frente a otros posibles competidores, obteniendo una alta factibilidad en su posible implementación desde el punto de vista del análisis financiero como de mercado. Consecuentemente este plan de negocios esta soportado por su aprobación en el  Programa de Aceleración de Proyectos de Emprendimiento e Innovación Empresarial de la Cámara de Comercio de Bogotá (CCB).