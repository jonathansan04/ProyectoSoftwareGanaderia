\subsection{Riesgos operacionales}

El término gestión de riesgos operacionales (ORM) se define como un proceso cíclico continuo que
incluye la evaluación de riesgos, la toma de decisiones de riesgos y la implementación de controles de riesgos, lo que da como resultado la aceptación, mitigación o elusión de riesgos. ORM es la supervisión del riesgo operativo, incluido el riesgo de pérdida resultante de procesos y sistemas internos inadecuados o fallidos; factores humanos; o eventos externos. \cite{gerencia_2022}.

\textbf{Riesgo:} Fallos en los sistemas de cómputo, tanto en hardware como software de la empresa.

\textbf{Contra medida:} Se optará por soluciones en la nube de pago por uso que dotarán a la empresa con características como la escalabilidad y la alta disponibilidad.

\textbf{Riesgo:} Daño o pérdida de información por errores humanos, causando Interrupción del negocio y fallas en los sistemas.


\textbf{Contra medida:} Implementar roles en todos los sistemas utilizados por la empresa, manteniendo las funciones de cada quien debidamente administrada.

\textbf{Riesgo:} La infraestructura no soporta el desarrollo de las aplicaciones necesitadas por la empresa.


\textbf{Contra medida:} Realizar una correcta estimación de necesidades y requerimientos tanto de recursos físicos, como lógicos y de personal para poder llevar a cabo de la mejor manera los productos finales.

\textbf{Riesgo:} Incapacidad de cumplir con los objetivos y plazos pactados al iniciar un proyecto nuevo.


\textbf{Contra medida:} Realizar un correcto y completo estudio de los proyectos nuevos que ingresen a la empresa, como también un análisis completo de las capacidades del personal contratado, para poder de esta manera cumplirle al cliente, en cuanto a funcionalidad y tiempos se trata.