\section{Análisis de riesgo}

Al crear una empresa, se requiere que los procesos de gestión y planeación tengan en cuenta realizar un análisis de riesgos, puesto que estos análisis nos permiten identificar los riesgos de distinto tipo que puedan afectar la empresa o a las que esta pueda estar expuesta, y por lo tanto, crear estrategias que permitan mitigar dichos riesgos o reducir la probabilidad de ser afectado por ellos. \cite{merchan}.

El Análisis de Riesgos significa examinar la magnitud y la índole de los posibles efectos negativos de la introducción propuesta, así como la probabilidad de que éstos se produzcan. Deberá identificar medios eficaces para reducir los riesgos y contemplar alternativas a la introducción propuesta. \cite{iucn}.

\import{./}{Content/12.AnalisisDeRiesgo/FactoresLimitantesObstaculos}

\import{./}{Content/12.AnalisisDeRiesgo/FactoresClaveExito}

\import{./}{Content/12.AnalisisDeRiesgo/RiesgosLegales}

\import{./}{Content/12.AnalisisDeRiesgo/RiesgosOperacionales}
