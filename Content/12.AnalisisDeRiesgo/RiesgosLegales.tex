\subsection{Riesgos legales}
La probabilidad de pérdidas o interferencias a un negocio derivadas principalmente de transacciones defectuosamente documentadas, reclamos o acciones legales, protección legal defectuosa de los derechos/activos de la empresa y/o desconocimiento normativo o cambios en la ley o su interpretación \cite{danielPerez}. Los riesgos legales también incluyen la posibilidad de un incumplimiento de regulaciones legales y/o un riesgo legal que se produzca por un conflicto de intereses \cite{empresa_2015}.

\textbf{Riesgo:} Protección de la empresa ineficiente.

\textbf{Mitigación:} Se llevará a cabo a través de políticas de la empresa, un excelente manejo en temas relacionados con propiedad intelectual, marca, buen manejo y disponibilidad de las bases de datos, administración en las inversiones, secretos industriales y garantías, junto al equipo encargado y especialista en el tema.

\textbf{Riesgo:} Reclamos y acciones legales.

\textbf{Mitigación:} Las asesorías por profesionales especializados en materia legal, se encargará de los procesos jurídicos en cuestión de reclamos, procedimientos, investigaciones y demás procesos administrativos, que eviten todo tipo de multas y sanciones, entre otros.

\textbf{Riesgo:} Uso de software sin licencia.


\textbf{Mitigación:} Las políticas de licenciamiento de software, proporcionan los lineamientos del uso de software con licenciamiento bien sea libre o de pago junto a su certificado, así como del personal encargado para coordinar estas tareas.

\textbf{Riesgo:} Incumplimiento de contratos.


\textbf{Mitigación:} Con la asesoría de personal especializado en leyes, se ejecutarán los contratos pertinentes en donde se evidencie de forma clara los requerimientos, servicios y demás procesos que sean parte del modelo de negocio y en software como servicio.