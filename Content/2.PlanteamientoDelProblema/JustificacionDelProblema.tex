\subsection{Justificación del Problema}
El desperdicio de alimentos es un problema económico, ambiental y social en todo el mundo. En Colombia, según un estudio del Banco de Alimentos de Bogotá, el desperdicio de alimentos ha llegado a 9,76 millones de toneladas, causando inseguridad alimentaria al 42.7\% de hogares \cite{BancoAlBogt}.

Por un lado, abarcando un punto de vista económico los alimentos que no se venden terminan en la basura sin generar ingresos. Según el Banco Interamericano de Desarrollo (BID), el desperdicio de alimentos en Colombia representa una pérdida económica de aproximadamente 45.365 millones de pesos colombianos al año.\cite{BID} . Por otro lado, desde la parte social, reducir el desperdicio de alimentos puede tener un impacto positivo en la población de bajos ingresos. Según la Organización Internacional del Trabajo, las desigualdades económicas han crecido en Colombia en los últimos años, lo que ha generado que la población con bajos ingresos tenga dificultades para acceder a alimentos frescos y saludables, solo en Bogotá el 54,2\% de familias padecen de inseguridad alimentaria.\cite{OIT}

Con la disminución del desperdicio de alimentos aumenta la disponibilidad de los mismos y el consumo de estos, así como las ganancias económicas en sus ventas.  
Para abordar esta problemática, se propone un plan de negocio enfocado en el aprovechamiento de alimentos basado en el caso de estudio de nómadas digitales, con el fin de ofrecer esos alimentos a un menor precio y evitar que se desechen en mercados y tiendas de Bogotá. La creación de una plataforma digital para la venta de alimentos a menor precio significa una solución innovadora y efectiva para reducir el desperdicio de alimentos en tiendas y mercados de Bogotá, al mismo tiempo que se ofrece una alternativa económica a los consumidores y se promueve la sostenibilidad ambiental y la justicia social.

Según la investigación realizada por Frewer Lynn docente especializada en la seguridad alimentaria de la Universidad de Newcastle, el uso de plataformas digitales es una estrategia efectiva para reducir el desperdicio de alimentos, ya que permite conectar a los comerciantes con los consumidores finales de manera más eficiente y efectiva.\cite{FoodScience&Tecnology}. 
