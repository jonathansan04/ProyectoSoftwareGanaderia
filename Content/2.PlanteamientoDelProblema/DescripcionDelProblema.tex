\subsection{Descripción del Problema}
Hoy en día las poblaciones vulnerables del país se enfrentan a la incetidumbre en cuanto a la seguridad alimentaria, los altos costos de los alimentos dificultan la accesibilidad a los mismos, multiplicando el hambre y aumentando los niveles de desnutrición. Según la Organización de las naciones unidas para la alimentación y la agricultura (FAO), la prevalencia de desnutrición total en la población de colombiana para el 2021, fue de 8,2\%, equivalente a 4,2 millones de personas \cite{FAO} siendo esta la causa de diferentes factores, uno de ellos, el desperdicio de alimentos.

Este desperdicio no solo impide que los alimentos sean aprovechados por aproximadamente 8 millones de personas al año \cite{DNP}, también se le responsabiliza del 7\% de las emisiones globales de Gases de Efecto Invernadero (GEI), según la Organización de las Naciones Unidas (ONU)\cite{MADS}.

La gestión de los excedentes alimentarios afecta directamente los niveles de los alimentos desperdiciados, brindando una oportunidad de servicio en la intermediación e implementación de sistemas de venta de alimentos a precios reducidos para los sectores de ventas de alimentos de uso diario como tiendas y mercados de plaza, la cual ha mostrado éxito en otros países de la mano de soluciones tecnológicas en el marco de los nómadas digitales \cite{Soup}, apoyando la disminución  del problema.

