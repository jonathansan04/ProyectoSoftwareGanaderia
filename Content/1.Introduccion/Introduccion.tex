\section{Generalidades}

\subsection{Introducción}

Con la tecnología, la información está cada vez más disponible y a la mano de todas las personas, ofreciendo y generando visibilidad a los diferentes problemas y situaciones vividas al rededor del mundo, uno de esos problemas ha sumado peso a lo largo del tiempo, la falta de concientización y la ignorancia sobre el cuidado de los alimentos ha contribuido al crecimiento de los niveles de pérdidas de los mismos. Para el 2016 en Colombia la pérdida y desperdicio de alimentos ascendió al 34\% del total de comida disponible en el país. Es decir, de las 28,5 millones de toneladas de alimentos que se podrían consumir al año, se desperdician o se pierden 9,8 millones de toneladas \cite{DNP}, el estudio más reciente realizado por la Unidad Administrativa Especial de Servicios Públicos indica que sólo en Bogotá se desaprovechan 1.228.000 millones de toneladas de alimentos al año \cite{UAESP}, estas alarmantes cantidades generan una oleada de dudas respecto a los procesos que se deben llevar para disminuir la pérdida de alimentos que viene en aumento.

En los procesos de ventas, el 28\% de los alimentos son desechado por los supermercados, tiendas de barrio y plazas de mercado \cite{DNP} involucrando una variedad de productos como frutas, verduras, cereales, carnes y hasta alimentos ya preparados que no lograron ser vendidos al final del día y que luego terminan en la basura al no haber más compradores, es por esto que en este documento se realizará el planteamiento de un plan de negocio para apoyar en el proceso de disminución del desperdicio de alimentos en mercados y tiendas de Bogotá, con el fin de llevar en conjunto de la tecnología una solución que aporte una nueva visión acerca de cómo abordar el problema desde el caso de estudio de nómadas digitales, y que al mismo tiempo logre convertirse en una oportunidad de transformación para que los alimentos 
que estén destinados a ser desechados puedan ser aprovechados y comercializados a un menor precio, apaciguando las preocupaciones desarrolladas a partir de los costos alimenticios.
\newpage